\documentclass[11pt]{article}

%  USE PACKAGES  ---------------------- 
\usepackage[margin=0.7in,vmargin=1in]{geometry}
\usepackage{amsmath,amsthm,amsfonts}
\usepackage{amssymb}
\usepackage{fancyhdr}
\usepackage{enumerate}
\usepackage{mathtools}
\usepackage{hyperref,color}
\usepackage{enumitem,amssymb}
\usepackage{xurl}
\newlist{todolist}{itemize}{4}
\setlist[todolist]{label=$\square$}
\usepackage{pifont}
\newcommand{\cmark}{\ding{51}}%
\newcommand{\xmark}{\ding{55}}%
\newcommand{\done}{\rlap{$\square$}{\raisebox{2pt}{\large\hspace{1pt}\cmark}}%
\hspace{-2.5pt}}
\newcommand{\HREF}[2]{\href{#1}{#2}}
\usepackage{textcomp}
\usepackage{listings}
\lstset{
basicstyle=\small\ttfamily,
% columns=flexible,
upquote=true,
breaklines=true,
showstringspaces=false
}
%  -------------------------------------------- 


%  HEADER AND FOOTER (DO NOT EDIT) ----------------------
\newcommand{\problemnumber}{0}
\pagestyle{fancy}
\fancyhead{}

\newcommand{\newquestion}[1]{
\clearpage % page break and flush floats
\renewcommand{\problemnumber}{#1} % set problem number for header
\phantom{}  % Put something on the page so it shows
}
\fancyfoot[L]{IE 332}
\fancyfoot[C]{Assignment submission}
\fancyfoot[R]{Page \thepage}
\renewcommand{\footrulewidth}{0.4pt}

%  --------------------------------------------


%  COVER SHEET (FILL IN THE TABLE AS INSTRUCTED IN THE ASSIGNMENT) ----------------------
\newcommand{\addcoversheet}{
\clearpage
\thispagestyle{empty}
\vspace*{0.5in}

\begin{center}
\Huge{{\bf IE332 Project \#2}} % <-- replace with correct assignment #

Due: April 28th, 11:59pm EST % <-- replace with correct due date and time
\end{center}

\vspace{0.3in}

\noindent We have {\bf read and understood the assignment instructions}. We certify that the submitted work does not violate any academic misconduct rules, and that it is solely our own work. By listing our names below we acknowledge that any misconduct will result in appropriate consequences. 

\vspace{0.2in}

\noindent {\em ``As a Boilermaker pursuing academic excellence, I pledge to be honest and true in all that I do.
Accountable together -- we are Purdue.''}

\vspace{0.2in}

\begin{table}[h!]
  \begin{center}
    \label{tab:table1}
    \begin{tabular}{l|p{2.2cm}|p{2cm}|p{2.6cm}|p{2cm}|p{1.2cm}|p{1.2cm}|p{1cm}}
      Student & Algorithm Development & Complexity Analysis & Implementation & Performance Analysis/ Testing & Report & Overall & DIFF\\
      \hline
      Arisa Kulkarni & 20 & 20 & 20 & 20 & 20 & 100 & 0\\
      Allie Ranaldi & 20 & 20 & 20 & 20 & 20 & 100 & 0\\
      Noah Morrison & 20 & 20 & 20 & 20 & 20 & 100 & 0\\
      Jonathan Papp & 20 & 20 & 20 & 20 & 20 & 100 & 0\\
      Emilio Pozas & 20 & 20 & 20 & 20 & 20 & 100 & 0\\
      \hline
      St Dev & 0 & 0 & 0 & 0 & 0 & 0 & 0
    \end{tabular}
  \end{center}
\end{table}

\vspace{0.2in}

\noindent Date: April 28, 2023
}
%  -----------------------------------------

\begin{document}

\addcoversheet


%ASSIGNMENT BEGINS HERE
%TABLE OF CONTENTS
\newpage
\tableofcontents


%MAIN TEXT OF REPORT
\newpage
%Introduction & Objectives
\section{Introduction \& Objectives}
The following report details the team's \textbf{hypothetical} optimization algorithms for adversarial attacks on a binary image classifier. Given a pre-trained binary convoluted neural network (CNN) image classifier that determines if a given image is a dandelion or grass, the team created five different machine learning/optimization algorithms to fool the image classifier. Each algorithm has a specific weight based on the expected performance of the given imputed image. The machine learning and optimization algorithms provide a foundation that can be used in the real-world application of AI infrastructure in many different organizational domains. \newline

\noindent Machine learning algorithms are used to teach computers specific tasks; in this project, that task is a binary classification of images. These algorithms go through a training period to learn the information directly from trial and error. Each iteration of training improves performance(MathWorks). The training code was given to the team to evaluate and decide the best course of action to trick the CNN. \newline

\noindent At a high level, the objectives for the team were to:
\begin{enumerate}
\item Create five different machine learning and/or optimization algorithms that each fool the classifier into thinking the image is not what it actually is.
\item The five algorithms will each give a probability of deciding if the image is a dandelion or grass. 
\item Based on testing, each of the five algorithms will be given a rank of 1-5 (5 being the best, 1 being the worst), and based on the rank, weights that add up to one hundred will be given to each algorithm where rank 5 has the highest weight but not enough to overrule the other algorithms.
\item The results from the five algorithms will go into the weighted algorithm that combines the weights for each algorithm to officially give a result of dandelion or grass.
\end{enumerate}


%Image Classifiers
\section{Image Classifiers}
The image classifier provided for this project is an example of a convoluted neural network (CNN). Once it is trained, the way a CNN works is by sampling small sections of an input image at a time and calculating the convolution of neighboring pixel values (ApokalypsePartyTeam, 2021). In other words, it calculates how different neighboring pixels are from each other to determine distinguishing features such as sharp edges that create a shape or sharp color contrasts that make an identifying pattern. It can then refer to its archive of training to compare where it has seen similar features and make its prediction of what the image is.

%Adversary Attack 
\section{Adversary Attack}
Adversary Attacks are a technique that attempts to fool models with deceptive data (Wiggers, 2021). The most common reason to use these attacks is to malfunction a machine learning model. The team's machine learning and optimization algorithms are an adversary attack. The algorithms present an image with inaccurate/misrepresentation pixel data to fool the classifier. Some of the team's attacks are in the form of noise; one type of adversarial attack is White Box Attacks which can access the model's parameters and obtain labels for the inputs provided (Analytics Vidhya, 2022). 

%Algorithm Explanation 
\section{Machine Learning and Optimization Algorithms}
\subsection{Fast Gradient Sign Method}
Fast Gradient Sign Method, or FGSM, was one method used for a sub algorithm. This method was chosen because it is specially designed to create adversarial images, which are images meant to confuse a convoluted neural network and cause a misclassification (Rosebrock, 2021). Not only does it confuse a CNN, but FSGM also causes a CNN to confidently make a mistake. It accomplishes this by computing a loss function from the gradients of the input image from the CNN. The FGSM then determines which input pixels contribute to the loss value the most, and adjusts them accordingly to create the adversarial image (Tensorflow, 2022). The result is a "white box" attack, or a new image that appears identical to the input image to the human eye, but confuses the CNN by adjusting the colors of pixels slightly. The following code shows the FGSM function as used by the group. 

\begin{lstlisting}[language=R, frame=single]
library(keras)
library(imager)

#This is for loading the model and the chosen image
model <- load_model_tf("C:/Users/Quesadilla/Documents/332/Project2/model/dandelion_model")
img <- image_load("C:/Users/Quesadilla/Documents/332/Project2/data-for-332.tar/data-for-332/data-for-332/data-for-332/dandelions/dandelion_yellow_flower_230715-1599547728.jpg")

#this is for getting the loss function to compute the gradient of
loss_fn <- function(y_true, y_pred){
  mean(k_categorical_crossentropy(y_true, y_pred))
}

#this computes the gradient of the loss function and returns a new image
fgsm <- function(model, x, y, esp = 0.01){
  grad_fn <- tf.GradientTape(loss_fn, model$trainable_weights)
  grad <- grad_fn(list(x, y))[[1]]
  
  x_adv <- x + esp * sign(grad)
  
  return(x_adv)
}
#this applies fgsm to the new image and runs it through the classifier
x_adv <- fgsm(model, img, 1, esp = 0.1)

pred <- model %>% predict(x_adv)

class_label <- class.ind2label(pred)
\end{lstlisting}

\subsection{Simulated Annealing}
Simulated Annealing algorithm takes a heuristic approach to optimization problems. It is modeled from the annealing procedure of metal working, where a metal is heated and then slowly cooled to make its crystal structure more stable. Similarly, in simulated annealing, the algorithm starts with a high temperature, which allows it to explore a wide range of solutions and then gradually cools down to a low temperature, which forces the algorithm to converge toward the global minimum. SA algorithm proposes an effective solution to this problem by incorporating two iterative loops, which are the cooling procedure for the annealing process and the Metropolis criterion. (Science Direct, 2017) As this method is pure optimization to search for the best pixels to change, the result is Gaussian perturbation function that edits the image to be fooled with the set pixels.
\vspace{0.2 in}
\begin{lstlisting}[language=R, frame=single]
#Gaussian perturbation function
apply_perturbation <- function(img, x) {
  # Reshape x to match dimensions of image
  x <- matrix(x, nrow = nrow(img), ncol = ncol(img), byrow = TRUE)
  
  # Add perturbation to image
  img_perturbed <- img + x
  
  # Clip pixel values to [0, 1] range
  img_perturbed[img_perturbed < 0] <- 0
  img_perturbed[img_perturbed > 1] <- 1
  
  # Return perturbed image
  img_perturbed
}

# Convert the image to a numeric vector
x0 <- image_to_array(img)

# Define the objective function
obj_fn <- function(x) {
  # Convert the numeric vector back to an image
  img <- array_to_img(x, dim_ordering = "tf")
  # Get the predicted class for the image
  class <- predict(model, image_to_matrix(img))
  # Return the negative probability of the target class
  -class[target_class + 1]
}

# Set the search space bounds
lower <- rep(0, length(x0))
upper <- rep(255, length(x0))

# Set the optimization control parameters
control <- list(maxit = 5000, trace = TRUE)

# Set the temperature and other parameters for the SA algorithm
temp <- 1.0
temp_min <- 0.01
alpha <- 0.99

# Run the SA optimization algorithm
result <- optim_sa(start = x0, fun = obj_fn, maximization = FALSE,
                   lower = lower, upper = upper, control = control)
 
\end{lstlisting}

\subsection{Particle Swarm Optimization}
Particle Swarm Optimization is an algorithm that generates an adversarial image that holds the ability to trick an image classification model. This is done by simulating interaction and movement between a group of particles in a controlled space to find the most optimal solution. This algorithm was inspired by the system of a flock of birds moving collectively to benefit from one anothers experiences (Tam, 2021). In this machine learning algorithm, the possible solutions are considered particles. The particles interact with each other by adjusting their movement velocity based on the experiences the particle itself has encountered and the experiences that it has witnessed from its neighboring particles. The velocity of the particle is updated with every iteration of a nested loop. With each updated velocity, the particle location and fitness are updated as well. The fitness function is a key component of the algorithm as it addresses the best-fit functionality of each possible solution. The function assigns numerical values of fitness to each solution based on the project objective. After evaluating the fitness scores of each particle, this machine learning algorithm is finally able to create an adversarial image that can be used to test the durability of the image classification model against potential adversarial attacks.\\

After loading the image classification model and test image in the R code and converting the images into an array of pixels, the group will create a fitness function to determine the functionality of the solution possibilities. This is done by the code displayed below:
\begin{lstlisting}[language=R, frame=single]
fitness_function <- function(particle) {
  # For generating the adversarial image
  adv_x <- x
  adv_x[,,particle == 1] <- adv_x[,,particle == 1] + epsilon
  adv_x[adv_x > 1] <- 1
  adv_x[adv_x < 0] <- 0
  
  # For evaluating the model prediction on the adversarial image
  pred <- model %>% predict(adv_x)
  
  # This returns 1 - predicted probability of true class as fitness score
  1 - pred[1, class_idx]
}
\end{lstlisting}

After initializing key variables such as the particles, velocities, personal and global best positions, and fitness scores, it is crucial to create code to update the velocities and position for each iteration. This will be used to calculate the fitness scores and generate an adversarial image with an ideal solution.
\begin{lstlisting}[language=R, frame=single]
# Define velocity and position updates
for (iter in 1:max_iterations) {
  for (i in 1:n_particles) {
    # This updates the velocity of the particle
    cognitive_component <- cognitive_coeff * runif(n_pixels) * (personal_best_position[i,] - particles[i,])
    social_component <- social_coeff * runif(n_pixels) * (global_best_position - particles[i,])
    velocities[i,] <- inertia_weight * velocities[i,] + cognitive_component + social_component
    
    # This updates the position of the particle
    particles[i,] <- ifelse(runif(n_pixels) < sigmoid(velocities[i,]), 1, 0)
    
    # This evaluates the fitness of the particle
    fitness <- fitness_function(particles[i,])
    
    # Update personal and global best positions
    if (fitness > personal_best_fitness[i]) {
      personal_best_position[i,] <- particles[i,]
      personal_best_fitness[i] <- fitness
    }
    if (fitness > global_best_fitness) {
      global_best_position <- particles[i,]
      global_best_fitness <- fitness
    }
  }
}
\end{lstlisting}


\subsection{Random Forest}
A random forest is a machine learning algorithm that combines the results of multiple decision trees to make a decision. In the context of this project, a random forest algorithm was tasked with deciding which pixels to change in an input image and then making a change, returning an adversarial image. On their own, decision trees are prone to errors like overfitting, but by using a random forest, the results are more accurate (Finnstats, 2021). By using this, a random forest can identify which pixels would result in an optimal change. 

\subsection{DeepFool}
The deep fool is an algorithm used in deep learning for generating adversarial examples. DeepFool is effective at generating small perturbations that can cause a deep neural network to misclassify an image. This makes it a useful tool for testing the robustness of machine learning models and identifying potential vulnerabilities. For binary classifiers works in the following way. 
\begin{equation}
    r_* (x_0) = - \frac{f(x_0)}{||w||_2^2}w
\end{equation}
The DeepFool algorithm computes the perturbation of r*(x) by having the output prediction of f(x\_0) divided by the L2-norm of the computed gradient w of the loss function, giving the scalar for the perturbation. This is then multiplied by the unit vector of w using L2-norm and finally has the sign inverted so the loss of the classifier f is increased, as shown in the equation. (Medium, 2022)
\vspace{0.2 in}

\begin{lstlisting}[language=R, frame=single]
# Define the DeepFool attack
attack <- deepfool(model, max_iter = 100)

# Generate an adversarial example
adv_img <- attack(img)
\end{lstlisting}
%Weighted Algorithm
\section{Weighted Algorithm}

%SOLUTION HIGHLIGHTS
\subsection{Solution Highlights}
After each of the smaller sub-algorithms had been created, testing was done on each of them with the given grass and dandelion data sets. Details of the testing are described in the appendix, and the results are shown below.\\
\begin{center}
\begin{table}[hbt!]
    \label{tab:table1}
    \begin{tabular}{c|c|c|c|c|c}
      \hline
      Algorithm & Grass & Dandelion & Grass (w/dummy) & Dandelion (w/dummy) & Results\\
      \hline
      Trained Model & 0.000 & 0.000 & 0.000 & 1.000 & 0.000\\
      Fast Gradient Sign Method & 0.959 & 0.974 & 0.333 & 0.667 & 0.967\\
      Simulated Annealing & 0.694 & 0.842 & 0.333 & 0.667 & 0.768\\
      DeepFool & 0.673 & 0.579 & 0.000 & 1.000 & 0.626\\
      Particle Swarm Optimization & 0.959 & 0.895 & 0.000 & 1.000 & 0.927\\
      Random Forest & 0.755 & 0.737 & 0.333 & 0.667 & 0.746\\
      \hline
      Weighted Algorithm & 0.980 & 1.000 & 0.444 & 0.555 & 0.990\\
      \hline
    \end{tabular}
\end{table}
\end{center}
The group used the results of the testing to determine the strengths of each sub-algorithm; i.e., which sub-algorithm provides the greatest probability of creating a successful adversarial attack on the image classifier. By ranking each sub-algorithm by its success rates, the team was able to create and assign weights to be used by the larger algorithm. The following weights were used. These weights were chosen as they sum to 100 voting points and allow the most successful sub-algorithm to have the strongest vote in the poll, but not such a strong majority that it cannot be outweighed. 
\begin{center}
  \begin{table}[h!]
  \begin{center}
    \begin{tabular}{c|c}
      \hline
      Ranking & Weight\\
      \hline
      1 & 28\\
      2 & 23\\
      3 & 19\\
      4 & 17\\
      5 & 13\\
      \hline
    \end{tabular}
    \end{center}
  \end{table}
\end{center}
The team also decided to keep the weights separate for the grass and dandelion models, as the ranking of the sub-algorithms differed between the two. Therefore, based on the results of the testing data and the pre-determined weights, the following voting weights were assigned.
\begin{center}
\textbf{Grass \space\space\space\space\space\space\space\space\space\space\space\space\space\space\space\space\space\space\space\space\space\space\space\space\space\space\space\space\space\space\space\space\space\space\space\space\space\space\space\space\space\space\space\space\space\space\space\space\space\space Dandelion}
\begin{table}[h!]
  \begin{center}
    \label{tab:table1}
    \begin{tabular}{c|c|p{2cm}|c|c}
      \hline
      Algorithm & Weight & & Algorithm & Weight\\
      \hline
      Fast Gradient Sign Method & 28 & & Fast Gradient Sign Method & 28\\
      Particle Swarm Optimization & 23 & & Particle Swarm Optimization & 23\\
      Random Forest & 19 & & Simulated Annealing & 19\\
      Simulated Annealing & 17 & & Random Forest & 17\\
      DeepFool & 13 & & DeepFool & 13\\
      \hline
    \end{tabular}
  \end{center}
\end{table}
\end{center}
The large algorithm takes the pixels that are suggested to be changed by the sub-algorithms and multiplies them times the number of votes that the particular sub-algorithm owns. Then it computes the number of pixels \textit{n} that can be changed based on the image size and user-inputted proportion of changed pixels. Finally, it selects the first \textit{n} pixels on the list of pixels sorted by votes, and alters those based on the sub-algorithms. The outputted result from the larger algorithm is then sent through the training model for each image to observe results.

%APPENDIX
\pagebreak
\section{Appendix}


\subsection{Testing/Correctness/Verification}
To test the algorithms, the group first observed the success of the training model itself without any adversarial attacks. The group tested both the grass set and the dandelion set, and kept results separate. This provided control group scores to compare the score of the sub-algorithms. The group then repeated the process for each data set with each sub-algorithm and tracked their success in fooling the model. The results are displayed in the table below.\\\\
Regarding the correctness of the algorithm, the group added dummy images to the data sets, such as a bike and a bus. The group then repeated the testing as described above to observe the ability of the model to detect these non-target images, and the effect to which each sub-algorithm alters the success of the model in detecting them. The results are displayed in the table below. After observing these results, the group decided that the binary quality of the model was creating unusable data. Since the model must identify an image as either grass or dandelions, it will produce a high percentage of a wrong selection with a dummy image. Furthermore, the trained model seems to think that most images are grass, which skewed the results.\\\\
To verify the precision of these models, sub-algorithms, and the larger weighted algorithm, each of the above tests was repeated twice more. This was in an attempt to account for the random seed to which the machines might be set.\\\\
After observing the success of each of the smaller sub-algorithms, the larger weighted algorithm was created, as described in the Weighted Algorithm section above. The same tests were applied to the weighted algorithm. Results are displayed below, with success rate determined as the proportion at which the model fails to detect grass or dandelions. The grass data set has 49 images and the dandelion set has 38, and each test was run three times. Therefore the grass data set is out of 147 and the dandelion set is out of 114. Three dummy images were used three times; therefore, each of these data are out of 9. Finally, the Results column is the average of the first two columns, since the group determined that the dummy image data was unusable.
\begin{table}[h!]
  \begin{center}
    \label{tab:table1}
    \begin{tabular}{c|c|c|c|c|c}
      \hline
      Algorithm & Grass & Dandelion & Grass (w/dummy) & Dandelion (w/dummy) & Results\\
      \hline
      Trained Model & 0.000 & 0.000 & 0.000 & 1.000 & 0.000\\
      Fast Gradient Sign Method & 0.959 & 0.974 & 0.333 & 0.667 & 0.967\\
      Simulated Annealing & 0.694 & 0.842 & 0.333 & 0.667 & 0.768\\
      DeepFool & 0.673 & 0.579 & 0.000 & 1.000 & 0.626\\
      Particle Swarm Optimization & 0.959 & 0.895 & 0.000 & 1.000 & 0.927\\
      Random Forest & 0.755 & 0.737 & 0.333 & 0.667 & 0.746\\
      \hline
      Weighted Algorithm & 0.980 & 1.000 & 0.444 & 0.555 & 0.990\\
      \hline
    \end{tabular}
  \end{center}
\end{table}

\subsection{Runtime Complexity and Walltime}
The runtime of each child algorithm varies because of the natures of each process. FGSM is a relatively quick process, as it is easy for a computer to find how much a pixel contributes to the overall loss using a chain rule and finding the required gradients (Tensorflow, 2022). Also, since the CNN is pre-trained, the model's parameters are constant and therefore reliably susceptible to the same attack, which means FGSM's only job is to fool a static model.\\
Due to all the sub-algorithms being complicated processes all of them have a large runtime complexity. As for walltime, the algorithms in total didn't take a long time to run(10-15s), but the group also couldn't get the results in the end due to different errors they encountered.

\subsection{Performance}
The above report is hypothetical. The group researched the five different sub-algorithms listed above and attempted to write code for each for weeks, but was ultimately unable to produce any successful code to fool the training model. The issues that prevented the group from implementing each algorithm are described below:
\begin{enumerate}
\item \textbf{FGSM:} An issue encountered with implementing FGSM that prevented further progress in the project with the limited time was a series of errors that were contradictory. In order to run the function as described in the group's research on the topic, a function called k\_gradients from the keras library needed to be used. However, a runtime error would say that this function was not supported and to use tf.GradientTape instead. When using this though, an error would say this function did not exist. 
\item \textbf{Simulated Annealing:} The issue encountered in simulated annealing was due to a series of errors that occurred at functions that the system didn't find inside the packages. The first was the array\_to\_image, and then the group also encountered it in img\_to\_matrix function when they tried to circumvent the previously mentioned function.
\item \textbf{Particle Swarm Optimization:} One issue encountered with the particle swarm optimization algorithm was the use of packages that did not function with the written code. The use of PSO requires a specific set of packages to solve the algorithm for the best fit particle. However, finding the correct combination of packages that would download efficiently and precisely was a challenge. For instance, the error stated that the package "magick" did not exist, but this package was needed to satisfy the PSO algorithm.
\item \textbf{Random Forest:} 
Issues with Random Forest occurred when trying to apply it to the fooling of the algorithm. Random Forest is usually used to train an algorithm to classify using multiple self-learning decision trees that sum up into a forest. This algorithm was a poor choice to use to fool the classifier since it's a learning algorithm rather than one that performs an adversarial attack on the image. 
\item \textbf{DeepFool:} In the DeepFool algorithm, the group didn't have the chance to run and attempt it due to the package of "adversarial" being a package that can only be found in RTools. When downloading RTools, the group then encountered the issue that RStudio did not detect the RTools packages. 
\end{enumerate}
Therefore, this report was written as if the group had successfully created each algorithm and then amalgamated them into one weighted voting algorithm.\\\\
Because the group was unable to produce successful code, the testing data described above is made up. However, this details the process that the group would have taken. Similarly, the rankings and weights that were applied to algorithm voting details the process that would have taken place. 

\newpage
\subsection{References}

Adversarial example using FGSM nbsp: nbsp; Tensorflow Core. TensorFlow. (n.d.). Retrieved April 27, 2023, from \url{https://www.tensorflow.org/tutorials/generative/adversarial\_fgsm} \\

ApokalypsePartyTeam. (2021, March 16). How to build your own image recognition app with R! [part 1]: R-bloggers. R. Retrieved April 27, 2023, from \url{https://www.r-bloggers.com/2021/03/how-to-build-your-own-image-recognition-app-with-r-part-1/} \\

Finnstats. (2021, April 13). Random Forest in R: R-bloggers. R. Retrieved April 27, 2023, from \url{https://www.r-bloggers.com/2021/04/random-forest-in-r/#:~:text=Random%20Forest%20in%20R%2C%20Random,to%20identify%20the%20important%20attributes.} 
\\

Husmann, Kai. “Flexible Optimization with Simulated Annealing.” R. Accessed April 27, 2023. \url{https://search.r-project.org/CRAN/refmans/optimization/html/optim_sa.html.} \\

Machine learning: Adversarial attacks and defense. Analytics Vidhya. (2022, September 1). Retrieved April 27, 2023, from \url{https://www.analyticsvidhya.com/blog/2022/09/machine-learning-adversarial-attacks-and-defense/#:~:text=The%20most%20common%20reason%20is,deceive%20an%20already%20trained%20model.}
\\

Morgan, Adrian. “A Review of DeepFool: A Simple and Accurate Method to Fool Deep Neural Networks.” Medium. Machine Intelligence and Deep Learning, May 2, 2022. \url{https://medium.com/machine-intelligence-and-deep-learning-lab/a-review-of-deepfool-a-simple-and-accurate-method-to-fool-deep-neural-networks-b016fba9e48e.}

Rosebrock, A. (2021, April 17). Adversarial attacks with FGSM (fast gradient sign method). PyImageSearch. Retrieved April 27, 2023, from \url{https://pyimagesearch.com/2021/03/01/adversarial-attacks-with-fgsm-fast-gradient-sign-method/#:~:text=Essentially%2C%20FGSM%20computes%20the%20gradients,image)%20that%20maximizes%20the%20loss.}
\\

“Simulated Annealing Algorithm.” Simulated Annealing Algorithm - an overview | ScienceDirect Topics. Science Direct, 2019. \url{https://www.sciencedirect.com/topics/engineering/simulated-annealing-algorithm.} \\

Tam, A. (2021, October 11). A gentle introduction to particle swarm optimization. MachineLearningMastery.com. Retrieved April 27, 2023, from \url{https://machinelearningmastery.com/a-gentle-introduction-to-particle-swarm-optimization/}
\\

What is machine learning?: How it works, tutorials, and examples. What is Machine Learning? | How it Works, Tutorials, and Examples - MATLAB &amp; Simulink. (n.d.). Retrieved April 24, 2023, from \url{https://www.mathworks.com/discovery/machine-learning.html} \\


Wiggers, K. (2021, May 29). Adversarial attacks in machine learning: What they are and how to stop them. VentureBeat. Retrieved April 26, 2023, from \url{https://venturebeat.com/security/adversarial-attacks-in-machine-learning-what-they-are-and-how-to-stop-them/}
\\
 
\end{document}